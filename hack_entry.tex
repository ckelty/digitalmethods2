\documentclass[10pt,letter,oneside]{scrartcl} 
\usepackage{setspace}
\usepackage[utf8]{inputenc} 
\usepackage[english]{babel}
%\usepackage{raleway} \renewcommand*\familydefault{\sfdefault} %% Only if the
%base font of the document is to be sans serif
\usepackage{graphicx} % Required for including images 
\usepackage{enumitem} 
%Required for manipulating the whitespace between and within lists
\usepackage[T1]{fontenc} % Use 8-bit encoding that has 256 glyphs
%\usepackage{natbib}
\usepackage{breakurl} 
\usepackage[breaklinks]{hyperref} 
\usepackage{pdfpages}
\usepackage{endnotes} \let\footnote=\endnote

\usepackage[style=authoryear]{biblatex} 
\addbibresource{hack_entry.bib}

\usepackage{authblk} 
\author[1]{Luis Felipe Rosado Murillo}
\author[2]{Christopher Kelty} 
\affil[1]{Berkman Center for Internet and Society, Harvard University} 
\affil[2]{Institute for Society and Genetics,
          Department of Anthropology, and Department of Information Studies, UCLA}
\renewcommand\Affilfont{\itshape\small}

\title{Hackers and Hacking}

\date{}

\begin{document} 
\maketitle 
\section*{Abstract} 

\doublespacing 

\section*{Introduction}

%Hackers, either self-proclaimed or identified as such by peers,  are everywhere
%to be found.  

Twenty five years ago, ``hacking'' was one of a range of underground practices,
associated with a particular politics and defining a set of individuals usually
characterized as adolescent, white males obsessed with computers.  Today,
literally anyone could call themselves a hacker, or any action a ``hack.'' In
recent decades, we have experienced the extension of the term to encompass many
ordinary technical practices in various domains, such as education, health care,
humanitarian response, farming, parenting, bodily modification, among many
others.  In Silicon Valley, companies like Facebook and Google elaborate a
``hacker way'' instead of (or in addition to) coding, engineering, or
entrepreneurship.  In public relation campaigns, hackers are described simply as
``doers'' since ``hacking just means building something quickly or testing the
boundaries of what can be done''
\parencite{funders_2016}. According to this very generous definition, we can all
be called ``hackers'', since we have been making artifacts for, at least, 2
millions years with the creation of mode-1 stone
tools \parencite{clark_world_1961}.

% so first we set up that there is an attentuation and proliferation of hacking,
% indicating that something is interesting people about it beyond the core.

We have also witnessed a real proliferation and globalization of hackfests,
hackathons, hackerspaces and gatherings around the symbol of hacking for a myriad of
purposes, including but not limited to the design of open hardware devices in
medical settings, the corporate-sponsored challenge of reinventing the soda
fountain with support from big companies such as Coca-Cola, the collective work
around public data-sets to fight corruption or help with questions of public
administration, or simply coming up with an inventive use of a product that is
about to be released in the market.  Historically, hacker gatherings---like
conferences in academic settings---have served the purposes of exchanging
knowledge and bringing small and fringe groups of computer aficionados together
to find peers, to share questions and findings that emerge when playing and
working with information and communication
systems \parencite{coleman_conference_2010}.  In the contemporary, many ``hacker
marathons'' have been organized by companies to identify programmers for hire,
offering comparatively small sums as prizes for new software or hardware
solutions which would necessarily cost considerably much more in research and
development.  A new business has been created around the work of head-hunting
for software engineering talent under the rubric of the ``hackathon,'' despite
its original connotation of a community-led sprint for developing technologies.

These facts suggest that the figure of hacking and hackers has become
fundamental to a contemporary technopolitical imaginary.  Whatever hacking is,
it is a very appealing figure and explanation for something.  Hacking is both a
label for something very general, while at the same time designating a very
specific set of practices and a very specific form of personhood.  In this
article we explore some of the ways hackers and hacking have been studied by
academics, as well as the forms of self-narration that different hacker groups
have themselves forged.  We argue that the subjectivity involved in cultivating
``hacking skills'' is not implicated in the range of things that can be called
``hacks''---or put differently not all hacks are perpetrated by hackers.  We
then ask what the relationship is between \emph{hackers} as a particular form of
personhood, and \emph{hacking} as a particular practice.  We end by suggesting
one possible way to decompose hacking into a ``stack'' of practices that can be
used to diagnose technical and political thresholds indicative of a mutation in
the ``topology'' of power in the world today.  Or to put it differently, there
is a reason why hacking seems to be spreading everywhere: because the forms and
affordances of technical and political power are themselves changing, and
hackers are at one forefront of experimenting with such mutations.

\section*{Stories of Hackers and Hacking}

From the early 1950's experimentation with communication and computing systems
to the present-day hacker activist initiatives in the Global North and South,
the narratives of hacking have been given different genealogies, supporting
different positionings with respect to whom and what counts as a legitimate
expression of hacking.  Most of these influential narratives have been provided
by journalists and self-designated hackers, and not by scholars.  Canonical
scholarly works are generally about the ``culture'' of computing---and not
specifically about hacking---which include those of Sherry Turkle
\cite*{turkle_life_1995,turkle_second_1984}, Diana Forsythe
\cite*{forsythe_studying_2001}; David Hakken and Barbara Andrews
\cite*{hakken_computing_1993}; Lucy Suchman \cite*{suchman_plans_1987}; Star and
Ruhleder \cite*{star_infrastructure_1996}; Stephen Helmreich
\cite*{helmreich_silicon_1998}; Joseph Dumit \cite*{dumit_picturing_2004}, and
Greg Downey \cite*{downey_machine_1998}. This literature has helped to address
the so-called ``myth of autonomous technology'' which presupposes a modernist
ontology which separates it from human culture and
society \parencite{latour_wehave_2008,winner_auto_1978}.  Pfaffenberger
\cite*{pfaffenberger_social_1992} advanced the anthropological study of
sociotechnical systems by refusing a separation between technology and
culture. In his studies of digital technology (the Usenet and the Personal
Computer as examples), he argued for the processual nature of technological
design as invariably embedded in cultural systems. With the exception of
Pfaffenberger and Turkle, little of this work is directly focused on hackers or
hacking as such, but nonetheless constitutes some of the most significant
ethnographic studies of computing in English-language.

The journalist Steven Levy is one of the most authoritative sources of the early
history, published in his mid-1980's book \emph{Hackers: Heroes of the Computer
  Revolution} \parencite{levy_hackers:_1984}.  Levy's work is exemplary in
offering early heroic narratives of the exploration of computer systems. It is
based on life-histories, tracing the origins of the hackerdom to the ``Tech
Model Railroad Club'' (TMRC) of the Massachusetts Institute of Technology (MIT)
of the 1950's. \emph{Hackers} has been translated into several languages and
accepted among distinct hacker communities worldwide.  Its narrative had the
performative force of instituting a return to the figure of the ``virtuous
hacker'' in the contemporary with its descriptions of the experience of early
hackers around research centers at MIT and Stanford, Northern California
collectives such as ``People's Computer Company'' and ``Homebrew Computer
Club'', and companies such as Apple Computer and Sierra Games.  Levy has also
popularized a positive definition of hacking as grounded in the ``hands-on
imperative'' and the ``hacker ethic.''  This ethic includes the commitment to
information freedom to facilitate technical exchange and to promote further
hacking; a rebellious attitude with respect to authority, centralization, and
control of computing infrastructures; and the idea that technical work could be
used to bring forth beauty and effect positive social change.  Similar stories
are told in the widely read books \emph{Where Wizards Stay up
  Late} \parencite{hafner1998wizards} and later in \emph{Hacker Ethic and the
  Spirit of Information Age} by the Finnish philosopher Pekka Himanem
(2001). The latter was calling attention to the subjective dimensions of a
cultivation that is distinctive of the computer hacker ascesis.

A more recent account of the early origins of hacking was given by the
technologist Phil Lapsley \cite*{lapsley_exploding_2013} in his book
\emph{Exploding the Phone: The Untold Story of the Teenagers and Outlaws who
  Hacked Ma Bell} which reconstructed the early history of ``phone-phreaking'',
the precursor of computer hacking which consisted in the exploration and
information sharing about phone systems.  Lapsley describes a genealogy which
connects the direct action of Yippies of the 1960's with exploration of
information and communication systems in the context of corporate control and
centralization of computing in the 1960's and 1970's.  His account calls
attention to a fundamental aspect of phone phreaking: a shared experience in
which the telephone became the very embodiment of curiosity, and the phone
network, a space for exploration, discovery, and socialization.

% This is an important period of transition of the past 50 years in which we have
% witnessed the passage from large-scale, military-grade computer infrastructures
% to distributed personal computing power with massive network integration in the
% Euro-American world. The consequences of this transition are key for the study
% of digitization: the transformation of cultural practices through the usage of
% digital technologies, difference in infrastructure (of computing networks),
% materiality (of computing devices), and virtuality (of archival and social
% memories).

A distinctive feature of hacker collectives resides in their effort of
self-organization around publications and gatherings.  Akin to other independent
groups, many phone-phreaking and hacker groups engaged in the practice of
self-documentation, with the publication of ``electronic zines'', manifestos,
and, in a few cases, with the enthusiastic adoption of an anthropological and
historiographic mode of inquiry.  Jason Scott (see Appendix) has emerged as a
self-appointed archivist of much of this material, maintaining extensive
archives of natively produced electronic documents, series of life-histories on
hacking, Bulletin Board Systems (BBS), and adventure, text-based games and much
else.  Eric Raymond, a well-known hacker and writer is also regarded by many as
a native anthropologist of hackerdom, having written on the culture and language
of hackers, and the moral norms associated with open source and free software
\parencite{raymond_cathedral_1999,raymond_art_2004}.  Raymond did much to preserve
and popularize the collaboratively produced document known as the ``Jargon
File'' which documented the rich language of early internet, usenet and hacking
terminology, and was republished as \emph{New Hackers Dictionary}
\parencite{raymond_new_1993} (see further reading).

Two major sociological and historiographic contributions in the literature
depicting the rise and fall of the "hacker underground'' of the 1980's and
1990's were ``Hacker Culture'' by the communications scholar Douglas Thomas
\cite*{thomas_hacker_2002} and ``Hackers'' by the sociologist Paul Taylor
\cite*{taylor1999hackers}. These two books are complementary in the sense that
they describe the underground hacker scene of the United States and the United
Kingdom in the period of popularization of hacker techniques and criminalization
of its practice.  Taylor's work is focused on the relationship between the
nascent computer security industry of the 1990's and the computer underground,
giving a fruitful description of the duality which is characteristic of the
underground lifeworld in which hackers are both the chaser and the chase, both
on the side of law enforcement and on the side of the curious hacker
collectives.  Douglas' work is particularly useful in describing the discursive
strategies in which the figure of the hacker as a unpredictable and
uncontrollable ``criminal'' was instituted -- having created a mythology around
the alleged superpowers of curious adolescents with access to a personal
computer, a modem, a phone line and certain access to information about flaws
and holes in computer and communication security.

The work of the science-fiction writer Bruce Sterling alongside others in
creating the ``cyberpunk'' literary genre led him to cross paths with hacker
groups in the United States in a key moment of its history: the late 80's and
early 90's in which the wave of ``crackdown'' on hacker collectives has become
widespread, especially in the wake of the new US Computer Fraud and Abuse Act
(CFAA). His book ``Hacker Crackdown'' narrates the story of police chase after
teenage hackers \parencite{sterling_crackdown_1993}.  Key in his depiction is
the argument of how misguided the police attempts were in framing computer
hacking as a serious criminal offense without understanding of the practice and
its consequences.  This period also saw the rise of several media darlings and
misfits, the most famous being Kevin Mitnick, who became the mainstream media
martyr after an incredible story of playing cat-and-mouse with the federal
police in the United States.  After being arrested, a mobilization of hackers
around the slogan ``Free Kevin'' took over the computer underground to clarify
the misguidance and overreach of prosecutors in Mitnick's case.  Many other
hackers' stories rose to prominence in this period, including the hacker
collective Legion of Doom (LOD) and their New York-based rival off-shoot Masters
of Deception (MOD).

In the 2000s, scholarly attention to hackers and related communities of computer
users picked up significantly, especially around Free and Open Source Software
on the one hand and online gaming on the other (the latter being too large a
field to touch on here, but see \textcite{coleman_review_2010a}).  Hackers in
free software projects formed the subject of work by Kelty, Coleman, Karanovic,
Hakken and
others \parencite{kelty_two_2008,coleman_coding_2012,auray_debian_2003,broca_utopie_2013}.
This work connected work on hackers directly to issues of intellectual property
and activism around it on the one hand, and also to questions about the liberal
underpinnings of Free Software and the question of the putative liberalism
and/or libertarianism of hackers themselves.  Golub and Coleman
\cite*{coleman_hacker_2008} argued most explicitly for refining our
understanding of the differences \emph{within} hackerdom by proposing several
``genres'' of hacking to get at distinctions with respect to the moral and
technical orders different hacker groups inhabit.

More recent publication and public debates around Free Software and hacker
communities has shifted the focus to questions of gender discrimination and
imbalance.  Alongside the work of feminist and women hackers around computer
collectives such as Systers, LinuxChix, Ada Initiative and Geek Feminism, new
publications have addressed the question of extreme disparity in Free and Open
Source projects where it is estimated that less than 2\% of the contribors are
women and other gender minories \parencite{ghosh_understanding_2005}.  Nafus
\cite*{nafus_patches_2012} has discussed the question of gender with respect to
the ways in which the organizing symbol of ``openness'' represents more than an
alternative to the intellectual property regime and a mode of managing the
collaborative efforts in software development.  According to the author, the
question of openness is accompanied with the insistence that gender plays no
role in software development, serving in fact to disguise the mechanisms of
exclusion of women and gender minorities from Free and Open Source projects.

% These mechanisms include but are not limited, to the maintenance of design
% decisions which demand previous technical knowledge, excluding newcomers; the
% creation of new expressions of masculinity that are exercised through public
% argumentation and defense of one's technical prowess, which are alienating to
% women and other gender minorities; and, last but not least, one of practical
% effects of ``openness'' which consists in denegating or outright denying
% ``difference'' for the sake of unfettered circulation of code, which has the
% effect of generating ``flame-fests'' when the question of gender is posed for
% debate among FOSS developers.

The most recent publication on the topic of ``hacking'' includes work by Michel
Lallemand on the topic of ``hackerspaces'' and the rise of the discourse and
practice of ``making'' \parencite{lallement}.  Lallemant has offered an ethnography
of the ``maker movement'' in the San Francisco Bay Area with a focus on the
question of the transformations labor and its reorganization with the creation
of independent spaces for collaborative work with digital technologies. The book
describes the community space ``Noisebridge'' in San Francisco which has been
one of the most influential autonomous spaces for the recreation of hacking as a
political symbol for self-organization and democratization of the access to
expert computing knowledge around the globe.

\section*{Hackers...}

One of the key insights of recent anthropology of hackers is how they represent
a distinctive elaboration of liberalism---especially in the domains of free
speech and the cultivation of a self directed towards freedom, autonomy, privacy
and other liberal values \parencite{coleman_coding_2012,kelty_two_2008,coleman_hacker_2014}. Hackers are
not uniformly libertarian or simply privacy advocates, but articulate a
relationship to technology with a different range of values depending on their
``genre'' \parencite{coleman_hacker_2008}.

Less work has been done on the way this relation to technology articulates with
different political and philosophical traditions outside the Euro-American
world, though a handful of people have pioneered such questioning.
 
% what to cite here: Lindnter and Murillo in China, Japan, Chan in Peru???. 

Complicating this question of the relation to technology, however, is the
question of what ``hacking'' is, how hacks are valued and assessed, and how they
are shared, learned, improvised, recorded and displayed, or circulated. 

Hackers are said to have a culture of perpetrating a particular kind of act
which can supposedly be distinguished from the actions of other kinds of people
in other communities (especially, those in bureaucracies, hierarchies,
corporations, or other hidebound organizations of the past).  The ``hack''
however, can be mobile and re-usable, it often, but not always, takes the form
of a tool or set of tools; it can be a one-off, round-about way of getting
something done, but it is often something that is re-usable, which serves as yet
another element for generalization.  Many ordinary practices of repurposing,
which in different contexts have different expressions, such as the
``gambiarra'', the Brazilian term for an improvisation or ``hack'', the
``juugad'', which expresses a similar improvisation of technical nature in
India, and the ``shanzhai'', which is a more specific form of repurposing in
mainland China of mobile devices.

% maybe also the Amish...  

In these cases, the ``hacker'' persona is troubled by the recognition that
``hacks'' are frequently borrowed, re-used, reconfigured and redeployed by
people who only subsequently come to think of themselves as hackers---or maybe
don’t do so at all.  Converse, depending on the context of occurrence and the
participants of the exchange, you could be called a hacker regardless of your
expertise, but solely on the basis of the technical feat you have performed.
Hacks are contextually visible to hackers, and depend on such assessment and
attribution to legitimately be called hacks.

Such a complicated identification of who counts as a hacker poses a dual
problem for anthropological research.  On the one hand, it creates a challenge
for making sense of personhood itself in a domain  where conventional idioms of
identity are being contested by the relation of the act and the person.  In
particular, hackers are often fond of rejecting the traditional credentials of
schools, employers, states or other entities that credential expertise in favor
of this recognition of the hack itself.  

But on the other hand, our anthropological attention is directed to the act
itself---``hacking''---and to the question of what it is and how it might be
studied anthropologically, or for instance, with the tools of science studies?
Are the actions of pirates, activists, criminals, spies and other such figures
``hacks''?  When do people call mid-level engineers, librarians, scholars and
designers ``hackers'' and when not?  Do all of these people share a particular
disposition of personhood, or do they share a milieu of technologies, material
and institutional forms at their disposal?  How can we disentangle the sudden
dispersal of hacking and hackers, in the sense of a wide circulation of devices,
persons, and discourses, around the world?

As we briefly explored in the previous section, there have long been competing
definitions of hackers.  Consider two distinctive definitions: one from the
Internet engineering community and the other from ``hacker underground.''  For
the former, a hacker is defined as a person ``who delights in having an intimate
understanding of a technological system,'' whereas his or her opposite, a
``cracker,'' is an individual who attempts to access computer systems without
authorization.  These hackers pride themselves on building complex systems out
of available parts, or on getting around an engineering problem in a simple and
elegant way--and they deride ``crackers'' as adolescents with only enough
knowledge to cause trouble.  As Coleman \cite*{coleman_coding_2012} details,
they are both highly individualistic in the Euro-American context, but oriented
towards a community of other hackers.

By contrast, for underground hacker collectives, hacking may signify
exclusively ``systems' penetration'' and exploitation as a display of technical
ability and mastery or, increasingly, for monetary gain.  For them, hackers are
people ``who gain unauthorized access to computer systems,'' a practice which
is evaluated in terms of technical aptitude and virtue, measured up against the
value of contributions in software code, information, and documentation.
``Exploits'' become objects of value that circulate---at one time only for a
kind of cultural capital amongst hackers, but increasingly today as part of a
robust market for ``Zero-day'' exploits (Soghoian xxxx).  Increasingly, these
forms of hacking are also well established in the military and defense world,
both defensively and offensively.

Members of the early Internet engineering expert communities claiming the hacker
status would strongly disagree with the definition given by or of the hacker
underground. Many of these communities were described in the origin stories of
Free and Open Source development as a ``natural attitude'' of pioneer computer
technologists.  They are strongly associated with research universities and with
a certain mythical understanding of scientific method and practice; they point
back to avatars such as the Digital Equipment Computer Users' Society (DECUS) in
the sixties, or SHARE, the IBM mainframe computer users'
group \parencite{akera2001voluntarism}, the MIT community around the Tech Model
Railroad Club, the Artificial Intelligence Laboratory operating system
development staff, and the early community around Berkeley Unix System
distribution, among many others.  The period which extends from the 1950's to
the 1970's is identified in the literature with the work of early, pioneer
hackers in pushing the boundaries of computing on many fronts: from hardware
hacking to personal computers, from new operating systems to video-games and
graphical user interfaces as we discussed above.

By contrast the ``hacker underground'' has its own mythical histories,
associated with the history of phone phreaking, bulletin board systems (BBS),
zines like ``2600'' and ``Phrack'', and movies like \emph{War Games} and
\emph{Hackers}.  These stories are more likely to reference the criminalization
of hacking under the US Computer Fraud and Abuse Act, and the fabled exploits
of people like Phiber Optik, Kevin Mitnick, Dark Dante, Erik Bloodaxe, and many
others.  While many Internet Engineers will point to Stephen Levy's
\emph{Hackers} as canonical, the hacker underground may point to Bruce
Sterling's \emph{Hacker Crackdown}.  Over the years, these two versions of
hacking have competed and collaborated: they cross paths at conferences like
DefCon, Chaos Computer Congress, and Hackers on Planet Earth, as well as at
major Free Software and Open Source events worldwide.  The \emph{cause célèbre}
of intellectual property activism amongst hackers--Dmitri Sklyarov---was
arrested at DefCon 9, but became a symbol for hacking as political
form---''code is speech'' \parencite{coleman_code_2009}. 

Both of these communities reject with derision the widespread use of the term
today in mainstream culture.  Facebook employees who ``hack'', ``brogrammers''
in Silicon Valley, and the generalized use of the term ``hack'' to mean ``do
something'' are seen as corruptions of various competing versions of a
tradition.   The Vitra ``Hack'' desk (See Fig \ref{fig:hackdeskimage}), for instance, performs an
attentuated and distorted version of hacking that emphasizes Silicon Valley
neoliberal start-up culture, an ``individually adaptable private sphere,'' and
a vision of flexibility in which the desks convert into sofas as part of a
longed for collapse of work, leisure and political authenticity, primarily
amongst white, upper middle class technology employees in Europe and North
America. 

\includepdf[pages={1,2},nup=1x2,scale=0.75]{images/Vitra_Hack}

\begin{figure} \centering
\includegraphics[scale=0.4]{images/Workspirit13HACKStudioAKFB3}
\includegraphics[scale=0.25]{images/HackDeskGrcic} \caption{The Hack Desk,
  designed by Konstantin Grcic, manufactured by Vitra.}
\label{fig:hackdeskimage}
\end{figure}

But the mainstreaming of hackerdom is just as likely to expose the sexism
and/or racism of past communities of hackers.  An ``elite hacker'' in an
interview to an influential hacker zine declared the demise of the hacker
underground with the xenophobic and mysoginistic observation that ``today it
is claimed that the Chinese and even WOMEN are hackers. Man, am I ever glad I
got a chance to experience `the scene' before it degenerated completely''
\parencite{phrackmag_2016}.  Symbolically violent in its own terms, this
observation indexes not only the ethnocentric and gendered lifeworld of most
hacker collectives, but the fact that hacking is no longer limited to the
virtual play-fight among Anglo-American suburban adolescents, pointing also to
questions of personhood with the evaluation of who gets to be considered a
hacker.

What is important to emphasize in these examples is not the adoption of one
definition or another, but the evidence of a series of disparate, conflicting,
and generative differences for the contextualization the hacking. This is the
evidence of an oscillation in respect to the value of ``hacking'' over time:
fluctuating from positive to negative moral valences, that is, from elite,
exclusive groups of technologists to online and offline communities marked by
the rhetoric of opennness and transparency, or, from what the communications'
scholar Douglas Thomas called the ``culture of secrecy'' of the hacker
underground in the height of the cold war to a ``culture of collaboration,
openness, and transparency'' as a shared utopian horizon in neoliberal times.

\section*{...or hacking?}

What is a hack?  And how is it different from leaks, breaches, exploits, or
other actions of information disclosure and circulation, both constructive and
destructive?

Leaking, for instance, has become associated with hacking in the last decade.
The furor around Wikileaks brought the fun-loving underground anti-collective
Anonymous to world-wide attention via its coordinated attacks on corporations
and governments.  Similarly, piracy: The May 15 movement in Spain was
spearheaded alongside protests of La Ley Sinde---an anti-piracy law widely
opposed by musicians and consumers alike.  These protests ultimately merged with
anti-austerity protests of the ``Indignados'' who shared key activists in the
North African revolutions, who were also involved in the movements organized by
Anonymous, who were often connected to actors in both May15 and Occupy, as well
as hacker activists of many groups including the group ``Telecomix''.  Taken
together, such actions have often been labeled ``hacktivism.''

Hacktivist movements, sometimes autonomous and sometimes part of larger
movements like Occupy, borrow tactics, technologies, slogans and ideas that both
explicitly and implicitly reference the ``hacking'' of Free Software, of the
Pirate Party, of anti-surveillance, pro-privacy hackers (like Tor) and copyright
reform movements like Creative Commons or activists fighting for Net
Neutrality. All of this has occurred at the same time that US and Israeli spies
were infecting Iranian nuclear power plants with the StuxNet virus
\parencite{zetter_countdown_2014}, and the NSA and GHCQ were cataloguing all of this,
and perpetrating some of the greatest ``hacks'' the internet has ever seen---and
this we know mostly because of the even greater intelligence leak of Edward
Snowden.

Hacking, as a practice, is not confined to hackers, but increasingly a practice
essential to the social fabric in surprising ways.  One might turn to ``practice
theory'' in its various forms to explore this---are hackers part of a
``community of practice'' constituded by hacks?  Is hacking an emergent
political expression of a particular sociocultural field, in Bourdieusian terms?
Are hackers heterodox actors in the context of a mainstream computing field,
wildcards in various professional domains, akin to the figure of Luther Blisset
or Guy Fawks who can come in and out of a mysterious identity?  Or is hacking
now mainstream itself?  

To speak of hackers as a community of practice, or of hacking as a field,
combines questions of technical and political cultivation with the
identification of a distinctive practice in order to suggest that the identity
of a hackers is based in the practice of hacking.  Indeed, such definitions are
common (we ourselves have proposed them---in the case of a ``recursive public''
for instance Kelty \cite*{kelty_two_2008}).  But this approach cannot accommodate the fact that
hacking and hacks have become more obsteperous and unruly in their circulation:
hackers fight hackers and trolls troll trolls and pirates steal from pirates
today.  To conflate the question of identity with the question of the practice,
political, technical or otherwise, would miss the mark.

Instead, there are a diversity of moral and technical orders inhabited by people
who are called or call themselves hackers. From the underground hacker
collectives to ``grey hat'' security researchers to spam-slinging criminal
hackers, to the hard-core free speech and privacy cryptography defenders; from
the die-hard Free Software activist to the business-oriented Open Source
evangelist; from the über-cool Northern European art hackers to the
goofy-but-terrifying Anonymous hackers, and so on.  As Coleman and Golub
\cite*{coleman_hacker_2008} point out, there is no single liberal ideology that
hackers adopt, but rather a range of ``genres'' in which any given individual or
group might operate, implying both the freedom and the constraints that word
signals.

But what is the substance of these moral genres?  We tentatively suggest here that these
``genres'' are not about identity or collective belonging or community.  They
are about specifiable practices to be sure, but they are bound up with new
kinds of technological affordances that make practices less dependent on
embodied skill than many other kinds of practice (e.g. glass blowing or cabinet
making) and as a result more mobile and modifiable.  The expanded field of
hacks, leaks, exploits, breaches, ops, online campaigns and so on form the very
substance of political life today in its complex entanglements with things,
protocols, perspectives, and sensibilities of digital technologists and
technologies. 

One reason to focus on the proliferation of hacking today---and its diverse
forms and expressions---is that it can help to pinpoint mutations in power and
knowledge, in a Foucauldian sense. Hacking, leaking, breaching create new tools,
as well as the collectives who make, maintain, teach and circulate those tools.
This has created a vastly expanded field of tactical power which challenge the
conventional ways in which Foucault's theory of power has been used---to go
beyond the simple, often epochally designated formations of sovereignty,
biopower/disciplinary power, and the governmentality of
neoliberalism \parencite{Macmillan2011a}.

In the last section, we imagine what a ``topology'' of power might look like if
hacking were disentangled into its component practices and tools, in order to
demonstrate that hacks are often a recombination of the elements of power, a
stretching of this topology in which action and response create tensions and
thresholds of power related to particular modes of hacking.  Approached this
way, one can better narrate the ``recombination of elements'' that Foucault (at
least later in his work) recognized as a way to stretch, transform or warp the
patterns of correlation that make up a given topology of
power \parencite{Collier2009a}.

``Hacker practices'' don’t always originate with hackers, nor do they confine
themselves to use by hackers.  Tactics or practices are picked up by computer
security scientists and researchers, security firms, police forces, political
campaigns, anti-piracy outfits, analysts of all sorts, as well as spreading
globally through networks of activists, hacker and maker spaces, legal firms,
musicians and artists, or consumers and pirates. Hackers go to work for Google
or Facebook, anti-piracy companies appropriate the tools of hackers and
pirates.  Power, in this sense, is about the recognition, appropriation,
recomposition, and redistribution of tactics and practices: not ideologies or
genres--but recomposable instances of power.  The technique of DDOS attack, for
instance, or the use of a cease-and-desist letter, the creation of
commons-producing copyleft licenses, or the use of the BitTorrent protocol can
all be identified with hacking, but they are not all of the same kind.

Building on the work of Coleman and Golub, we offer a brief proposal for
a set of distinctions that might make sense of these differences, and generalize
the generic categories they offer in order to expand the analytic
possibilities.  We describe this decomposition and give a handful of examples,
but cannot do more than suggest a possible future for anthropological studies of
hacking.  


...

The first three practices listed here (invention, inversion, and figuration) are
quasi-direct forms of action; they are accessible to anyone and do not require
large investments of money or stable organizations to mobilize.  The last two
(regulatory action and enforcement), however, are often more second-order or
``representative'' in that they often require physical, financial or
organizational resources and a certain scale and depth of involvement to
perform.  But we suggest that even these two forms have a ``hackish'' character
in the contemporary.

% I am not sure what to do with this "accessible to anyone"...  when the
% distinction here is "signficant investments of money or organization"

% CK: I tried to clarify this a bit above.



\subsection*{Invention}

Invention is the broadest possible meaning of hacking. It includes building and
making things, and not only material things, but especially so-called
``immaterial'' products such as software or organizations.  Invention might
include those practices that arise out of a lack or out a sense of need of some
sort, and at least in the conventional language of research and development it
comes with extensive planning, study and investment of time and money.  But as
``hacking'' invention often implies a possibility based in the existence of
multiple existing tools and components.  Creating software, hacking together
hardware components or starting up an organization are all practices of
invention in so far as they are carried out to solve a problem or respond to a
lack that makes certain possibilities clear.

Invention in the era of hacking has become much simpler and cheaper, as
toolkits, frameworks, patterns, languages and other easily reusable and either
cheap or free to use tools create a material culture of ready-made, accessible,
often lego-like parts.  Easy access to tools is by now an ethic (as in the case
of the Whole Earth Catalog,``Access to Tools'' described by Turner
\cite*{turner_counterculture_2006}) of mutual aid and instruction, and an
appreciation for both the DIY possibilities of our world, and sometimes a
respect for (and desire to contribute to) the coordinated engineering necessary
to bring them into being.

Furthermore, the practice of invention implies an affinity based in shared
understandings of how things work.  Whether that is a classic engineering
culture (based in University and/or corporate practice), a DIY geek culture, a
UNIX culture, a glass-blowing culture \parencite{o2005embodied} or whatever,
invention demands not only skill but the capacity to recognized others with more
or less the same skills, habits of practice and commitments to certain kinds of
technological or material choices \parencite{Sennett2008}.  It is this version of hacking
that is most clearly identified with, for instance, the Internet Engineering
communities described above, but also because of its generally positive valence,
the aspect which is emphasized by Silicon Valley start-ups.

\subsection{Inversion}

Inversion is a term meant to signal a practice that---at least under the label
of ``hacking''---is often insufficiently distinguished from invention.
Inversion is the kind of hacking that involves finding a way to use existing
tools or technologies to achieve something they were not meant to do.  Under
this aspect, exploits of vulnerabilities, using systems against themselves,
inverting the intended purpose of a system, or remixing something for purposes
of critique, parody, ridicule or something more practical
\cite{galloway2007exploit}.  In terms of hardware it includes the practices of
modding and customization; in terms of software it includes the practice of
finding and exploiting weaknesses in software (for good or evil intent) or
recombining software elements in a surprising and clever way in order to achieve
a new goal.

A famous example of inversion is neither hardware nor software but a so-called
``legal hack'': the General Public License (GPL), which uses existing statutory
copyright law to accomplish something it was not intended to do.

Inversion generally assumes institutional structures or infrastructures with a
certain transparency (one must be able to see how something works in order to
exploit it).  Some tactics exist for inverting the non-transparent---leaking,
such as the actions of Wikileaks or Snowden might be considered an inversion, as
might some forms of reverse engineering of secret techniques or technologies.
For hackers of the ``hacker undergound'' ilk, a whole range of tools exist for
attempting to break technologies, either to gain access, or to control them (the
creation of bot nets, zombie servers etc. 

Inversion also implies faith in engineering and in the rule of law: for
something to be inverted is not the same as for it to be destroyed or disabled.
GPL licenses work because copyright law is legitimate and enforced.  Remix or
recombination of software is done because the challenge is often to make
something work better or meet higher standards of practice or security.  By
extension, the faith is that things can always be and become better; only
imperfect things can be hacked--- inverted---in this sense.

Similarly inversion implies patterns of regularity that can be exploited---the
very thing upon which the practice of invention often depends as well.
Inversion works upon certain technological or organizational preferences shared
amongst a large group of people---for instance the use of SCADA control systems
amongst process engineers who design large industrial plants allows for hackers
to imagine exploits with powerful effects on the material world.  Similarly,
entrenched social behaviors are often exploited in ``social engineering'' hacks
that rely on the regularities of organizational design and human behavior.

\subsection*{Figuration}

It is also clear to many that not all hacking is restricted to technological
manipulation. We sugest \emph{figuration} to capture the more traditional and
recognizable forms of political action: rhetorical persuasion, ideological
argument, political advocacy, etc.  Classic descriptions of the functions of the
public sphere tend to characterize it as an issue of sovereignty---the ability
of ``the people'' to speak to and force changes amongst established domains of
power like the state, the church, the military,
etc. \cite{anderson2006imagined}.  All of the tactics involved here are about
visibility and the legitimacy of a political process.  In eras or locations
where there is nothing like a legitimate democratic government, tactics of
figuration are often ineffective or irrelevant.

To speak of figuration as a mode or component of ``hacking'' today, however, is
to recognize that such classical forms of discursive and persuasive power are
also used as forms, and in response to, hacking: operations by anonymous, for
instance, can be restricted to the widespread circulation of messages, videos
and manifestos. The protests against SOPA and PIPA in 2012 were largely
traditional responses by hackers (and others) to an attempted regulatory action.

Materially speaking, figuration depends on shared communication structures. Very
often today, hacking can take the form of \emph{inventing} or \emph{inverting}
communication practices as part of or in response to practices of figuration.
In many ways, open government data advocates of the last few years are demanding
that longstanding institutional structures (such as public hearings or requests
for comments) be hacked in order to make government similar or more responsive.
To do this they rely on a figuration of government as slow, non-responsive,
elitist or rigidly bureaucratic.

Substantively (at least in the domains of intellectual property and information
technology) practices of figuration have been heavily focused on issues such as
privacy, transparency, free speech, freedom to operate (or innovate) and network
neutrality.  These issues subtend a long and rich discourse that includes both
scholarly debates and conventional understandings of these concepts and their
value to our lives.  Examples of ``hackish'' figuration include the Electronic
Frontier Foundation, which was created in the context of ``hacker crackdowns''
of the early 1990's to articulate a discourse on the importance of defending
civil liberties online, primarily organized as a fund to legally defend hackers
from prosecutorial overreach.  Similarly, the death of Aaron Swartz provide a
figure of openness and open access as vital global struggles to which hackers
can and should contribute.

\subsection*{Regulatory action}

Regulatory action is not a form of hacking, but in many ways it's most important
opposite.  Nonetheless, a hackish attitude towards regulatory action has emerged
as a possibility---the strategic attempt to regulate (either formally as part of
goverment action, or informally through other means) represents another tool in
the hacker kit.  The language of regulation-as-control was most clearly
articulated in terms of hacking by Lessig's famous work \emph{Code, and other
  laws of cyberspace},
which argued that many different kinds of things regulate behavior (law, morals,
architecture) and are amenable to change to different degrees \cite{lessig2000code}.

At it's most general, regulatory action includes any form of policy change
intended to introduce, maintain or extend control.  State-based forms of
regulatory action are the most obvious and familiar but large corporations also
engage in regulatory action of particular kinds routinely---especially those
industries who control networks or technologies in widespread use.

Regulatory action almost necessarily implies large bureaucratic organizations of
a classic Weberian type---rule-based, hierarchically ordered, and subject to
regimes of oversight and transparency.  As a result, regulatory action is
relatively rare, and comparatively complicated to carry through.  The tactics of
invention, inversion and figuration are often oriented towards influencing,
responding to or disrupting regulatory action of various kinds---the 2012 case
of protests against SOPA/PIPA being a clear case; another case would be
Operation Payback conducted by members of Anonymous against the global credit
card companies Mastercard and Visa, who had engaged in the regulatory action of
systematically blocking PayPal donations to the Wikileaks organization.

% Regulatory action also implies a professional cadre acquainted with the rules
% and procedures of these organizations, oriented towards professional norms of
% democratic governmental practice, even if in practice such norms are honored in
% the breach or routinely subjected to the power of external interests.
% Nonetheless, regulatory action is something that is routinely pursued
% strategically by different groups---a corporation lobbies congress, one groups
% sues another or asks a court for an injunction; lawyers send letters; states
% create new roles and appoint people to them (e.g. IP Czar); courts mete??? out
% punishment, etc.

Inversion often borders on regulatory action when it is used strategically to
achieve something in the interests of a particular entity, but so too does
figuration, which is often the tactic most often employed to support or protest
a proposed legal change (both were used in the case of SOPA/PIPA).  Cases of
significant interest include those where regulatory action look more like cases
of inversion---i.e. where a legal action, for instance, is used to threaten,
intimidate or censor a particular group. The injunctions filed against
MegaUpload and Kim Dotcom, for instance, are represented as mere enforcement of
the law, but in reality represent the effective organization of state power by
industry organizations like the MPAA and the RIAA.

\subsubsection{Enforcement(s)}

Finally, there are practices of enforcement. Often only implicitly or
metaphorically included in a Foucauldian analysis (discipline is held to be
more insidious and more profound kind of force, a 'government of the soul' for
instance).  While classic displays of sovereign power are relatively rare
(helicopter raids by anti-terrorist forces on New Zealand mansions of
flamboyant wannabe hackers notwithstanding, in the case of Kim DotCom), they
remain a central tactic in the repertoire of power, and they are by no means
restricted to a state and its soldiers and police forces. 

Many forms of enforcement are widely available today.  A range of tactics and
practices of which the DDOS is only the most common are not confined to any
particular segment of society, but available to anyone who might, for instance,
download and install a copy of the Low Orbit Ion Cannon (or its predecessor,
organized by the Electronic Disturbance Theater group which in 1994 created an
application to flood the Mexican government website with requests to send a
message in support of Zapatistas).  Legal tools like cease-and-desist letters
are also routinely used as a tactic of force; even patent litigation can be
figured this way when conducted by so-called patent trolls.  Networked-based
forms of disruption are very often simultaneously tactics of invention or
inversion—coming into existence in response to the actions of states or
corporations.  Piracy and cracking generally might be said to move from being a
tactic of inversion when a vulnerability in a copy-protection scheme is merely
demonstrated (e.g. Sklyarov demonstrating holes in the eBook Reader) to a tactic
of force when that vulnerability is routinely exploited.

\begin{figure} \centering \includegraphics[scale=0.5]{images/protocolstackV2}
\caption{Stack of Power: an analytic decomposition of hacking
practices and how they might relate} \label{fig:ProtocolStack} \end{figure}

These five practices fit together as one possible description of how ``hacking''
is part of a topology of power today.  The software programmer's metaphor
of a ``stack'' is useful here---in common parlance it refers to a frequently
deployed but heterogeneous collection of tools (e.g. the ``LAMP'' stack is the Linux
Operating system, the Apache Web Server, the MYSQL database and one or more of
PHP, Perl or Python programming languages).  Such tools interface with each
other, and can in some cases come to depend on one another (nothing would happen
without an operating system in place, but there are many variants available).

The image of the stack we employ here, however, is meant to provide a basic map
of push and pull, of action and reaction, or of provocation and response.  We
intend it to be used to help diagnose certain technical or political thresholds
in the recent past of hacking (and we take it as axiomatic, as anthropologist, that
hackers are historically situated subjects who experience these thresholds in
their own lives and practices).   

Take for example, the case of Napster, which for many hackers around 2001,
represented a significant technical and political threshold.  A technical
threshold because it was an inversion of the normal system of distribution of music.
It used simple available technologies (the MP3 format, the ability of networked
computers to share files, and a simple interface) to quickly and more or less
elegantly demonstrate how the massive, quick and easy sharing of digital musical
files could be accomplished, almost overnight.  But it was also a political
threshold because just as quickly, it was shut down by a regulatory action
initiated by the RIAA---a clear demonstration of the continuing power of law
over the possibilities of technology.  There followed a war of figuration over
individuals figured as pirates by the industry, and as music lovers sharing
music by the proponents of intellectual property reform and technological
progress.  In response to this threshold, a technology like BitTorrent could
emerge as something engineered to get around the law---not simply a clever
invention in its own right, but one that created around certain legal limits. 

Napster was not, however, by any account at the time, a work of hackers.  It was
not free or open source software, Sean Fanning is often figured more often as
huckster than as hacker, and it does not figure in most narratives of the
history of hacking.  And yet the kinds of practices involved---ranging from the
immediate creation of the software, the use it made of available tools, the
figuration of the intellectual property wars, and the tide of
enforcement---take-downs, lawsuits, cease and desist letters and DMCA
notices---that followed are clearly central to the technical, political and
moral imaginaries of most hackers.

A similar diagnosis can be made of other cases: the creation of the Electronic
Frontier Foundation as a response to the CFAA, and the subsequent spread of
underground ``security'' research as a domain of legitimate hacking.  Or the
improvement, spread and embrace of Tor in response to the political movements of
2011 (as well as its use for criminal purposes, as in the case of the Silk Road
story), and the revelations of NSA spying. 

\section*{Conclusion}

To return, therefore to the open questions around an anthropological study of
hackers and hacking, we conclude with the following observations.  First, we
suggest that the study of hacking as a practice of ethical and technical
enskillment must move beyond the dominant concerns with privacy, autonomy and
security which remain the dominant narratives by hackers themselves and those
who study them.  Such narratives are part of an imperial imperative of digital
technology to transform sociotechnical difference through contact into its
resemblance (or poorly made copy).  Anthropological studies of the global
circulation of digital technologies and expert technologies have demonstrated
the naturalization of Euro-American underlying assumptions in digital design:
from user interfaces to data models; assumptions of usefulness of particular
technologies based on the prestige of their place of creation; and the
imposition of particular projects from centers of production to disconnected
peripheries of the global South.

% --> I would add here: how is the global difference in personhood of hackers
% related to the global accessibility of hacks?  What are examples of the
% reverse process, where the actions of global hackers (say, book pirates in
% russia or activists in Tunisia, or Anonymous' attacks on ISIS) have an impact
% on mainstream practices?

The recent past of hacking and hackers clearly overlaps with other
practices--those of piracy, activism (whether labeled cyber-activism or
hacktivism), trolling, leaking and breaching.  Such practices should be
understood not simple as forms of hacking, but as part of topology of power
which locally stable, but historically changing.  History favors the
stabilization of forms of power, but the manifest speed and ease with which the
contemporary topology can be stretched and deformed suggests a perpetual
oscillation or turbulence, limited only by the energy and enthusiasm that can be
committed to the various practices of invention, inversion, enforcement or
regulation.

% -- 

% {\bf Power imbalances, gender and ethnic troubles.} One of the key markers of
% difference among computer experts concerns gender and the exercise of
% particular forms of masculinity in heated arguments around technical issues. It
% is key for displaying structured relations of power and practices of value and
% worth assessment within hacker communities.  Questions of power and difference
% refer to the nature of sociotechnical ties within expert communities. In this
% sense, hacker collectives imagine themselves as meritocratic orders for having
% instances in which merit is recognized and rewarded through vetted collective
% mechanisms, but it is also oblivious to unequal starting points for the
% development of tech skills for participation in hacker collectives, such as
% Free and Open source projects. The gender question in digital communities is
% open for empirical study since most of the literature on hacking has dedicated,
% at most, footnotes to it. The question of power differentials (including
% socioeconomic and ethnic differences) is a broader issue concerning the place
% and role of hackers in various domains of computing: hobbyist, academic, and
% corporate.

%--> so is this about a "power imbalance" really?  To me it seems to be more
%about inequality and injustice than about power.  It's just as likely that
%gender inequity exists amongst powerful people (Silicon Valley venture
%capitalists) as amongst the weakest (local bike cooperative in my
%neighborhood).  So I don't see this as an "open question"--- or at least, it's
%open only in the sense that we don't address it in any novel way here...

% LF: Power imbalances in the sense of power differential within these
% collectives we study with... open question because, especially with respect
% to gender, the literature does not have much to say about it, except for a
% few articles.

% CK: I think this claim is something to be made in the literature review--it's
% not like their aren't people studying this problem, or asking how too, just that
% there is very little consensus in the literature.  So maybe just add a note to
% the last paragraph (where we cite Nafus) about how this is an under-studied
% and poorly theorized problem?

% --

% {\bf Informational and intellectual enclosures.} Another important aspect of
% hacker politics has to do with the struggle against informational enclosures
% through the practices of invention and inversion we mentioned before. The
% rebelious attitude of most hacker collectives with respect to hoarding of
% information and computational resources stems from pioneer hacker collectives,
% for whom the scarcity they experienced which led them to exploit available
% systems and force their way into computer laboratories (that were heavily
% guarded, given their absurd cost of computing machinery at the time). The
% struggle against the monopolization and control of computing has advanced in
% many areas -- such as game piracy, information security, and Free and Open
% Source software development -- creating communities whose members are
% identified as hackers but whose practices differ in respect to the moral and
% technical orders they inhabit. One of the most vocal hacker community in the
% dispute against the advancement of the Intellectual Property regime (IPR) has
% been the Free Software community, which has consistently for the past 30 years
% created mechanisms, both legal and technical, to subvert the logic of copyright
% and create alternatives to the patent systems in many legislations across the
% world.

% -->  I feel like this one is in line with the five practices above, and we
% could describe it in those terms:  IP Law and Invention are counteracted by
% the "inversion" of the GPL and the "figuration" of an informational
% commons...  DMCA is a "regulatory response"-- jailing Sklyarov is one kind of
% enforcement, creating BitTorrent is an invention in response to the
% crackdown... etc.  

% LF: Yes, I feel that we are repeating ourselves here, we can just copy and
% paste this part onto the discussion of figuration and regulatory response.

% {\bf Algorithmic Agencies.} For anthropological studies of digital platforms
% and cultures, the ethnographer must attend to the duality with respect to the
% submission of users to algorithmic agency and the fact that certain human
% actors are active agents in what has been called the ``web of computing''
% (Scacci and Kling) which includes many layers of subsystems that compose the
% domain of computing as a whole, not as a discrete-entity type).  Anthropology
% is well suited to study algorithmic agencies with ethnographic approaches to
% the lifeworld of research and development laboratories where computer systems
% are designed, discussed, re-designed, and implemented. Important contributions
% in this area have given by the pioneer ethnographer Diana Forsythe in her study
% with AI researchers building ``expert systems'' for the medical field, as well
% as Greg Downey in his work with engineers working with Computer-Aided Design
% technologies, Harry Collins in his sociology of technical expertise with the
% study of AI-based expert systems, and Lucy Suchman with interactional domain of
% experience involving engineers and machines at Xerox. 

%-->  too far afield I think.  There is room to ask, however, "how is what we
%are talking about here---hacking as personhood and practice---related to
%interests in big data and algorithms (e.g. Kate Crawford, Tarleton Gillespie,
%Natasha Schull, Frank Pasquale etc). 

% LF: I wouln't mind deleting this.

{\bf Collaborative Horizons.} The digital is not a panacea for the problems of
collaborative research in anthropology and the human sciences at large, but it
can certainly help if digital technologies are redesigned to further promote
collaborative engagements between ourselves and ourselves with the communities
we study with. Digitization of field research carries potential benefits with
respect to the ease of archival and data sharing, allowing for longitudinal and
extensive comparative studies. But is also carries serious issues of data
privacy and anonymity as most researchers (in both the sciences and humanities)
are not well equipped and informed to handle problems of information security.
Hacking, in this regard, is an important source of practices and workarounds,
both in the sense of invention and inversion, to deal with questions of
information security, but also sharing and remote collaboration across
transnational lines of exchange.



%-->  This is a good thing to conclude with--- essentially, "are
%anthropologists hackers, should we be?"

% this is what people asked me at the New School: how can we hack anthropology.
% can we pause for a minute? I would rather end with a strong articulation of
% practices: invention, inversion, figuration, etc. with respect to the open
% questions for future studies of hacking ...


% CK: Agreed... we can put this in the conclusion as a provocation, but I don't
% think we should be promoting the simple adoption of hacking by
% anthropologists--it's a bit too easy, and academia is too slow to keep up with
% the changes, both technical and political, I think.





\section*{Further References}

\begin{itemize}
\item  Original MIT Jargon File, kept by Paul Dourish:
  http://www.dourish.com/goodies/jargon.html 
\item  Extended Jargon File at Eric Raymond's site:
  http://www.catb.org/jargon/html/
\item Jason Scott's textfiles:
  http://textfiles.com/ 
\item Hackerspaces wiki: 
  http://hackerspaces.org
\end{itemize}

%\bibliography{hack_entry} \bibliographystyle{agsm}

\printbibliography 


\end{document}

