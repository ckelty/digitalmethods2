\documentclass[10pt,letter,oneside]{scrartcl}
\usepackage{setspace}
\usepackage[utf8]{inputenc}
\usepackage[english]{babel}
%\usepackage{raleway}
%	\renewcommand*\familydefault{\sfdefault} %% Only if the base font of the document is to be sans serif
\usepackage{graphicx} % Required for including images
\usepackage{enumitem} % Required for manipulating the whitespace between and within lists
\usepackage[T1]{fontenc} % Use 8-bit encoding that has 256 glyphs
%\usepackage{natbib}
\usepackage{breakurl}
\usepackage[breaklinks]{hyperref}
\usepackage{pdfpages}
\usepackage{endnotes}
\let\footnote=\endnote

\usepackage[style=authoryear]{biblatex} 
\addbibresource{hack_entry.bib}

\usepackage{authblk} 
\author[1]{Luis Felipe Rosado Murillo}
\author[2]{Christopher Kelty} 
\affil[1]{Berkman Center for Internet and Society, Harvard University} 
\affil[2]{Institute for Society and Genetics, Department of Anthropology, and Department of Information Studies, UCLA}
\renewcommand\Affilfont{\itshape\small}

\title{Hackers and Hacking}

\date{}

\begin{document} 
\maketitle 

\section{Abstract} 


%% TODO: fix abstract, fix intro

\doublespacing 

% From Gertaud:

% a) Starting point and relevance of the concept for cultural analysis, specific contribution in respect to digitization phenomenon.
% b) Detailed presentation of the concept, discussion of key texts and (brief) history
% c) Example of a study which demonstrates in an exemplary way the analytical potential and practice
% d) Reception, critique and scope of the concept, potential further developments or trends
% e) Reference to other resources contributing to the understanding of these concepts such as blogs, wikis etc. (organized as a box at the end of the contribution)


\section{Introduction}

Hackers are everywhere you look, these days.  Fifteen years ago, ``hacking'' was one of a range of underground practices, associated with a particular politics, and defining a set of individuals usually characterized as adolescent white males obsessed with computers.  Today, literally anyone can call themselves a hacker, or any action a ``hack.'' In recent decades, we have experienced the extension of the term to encompass many ordinary technical practices in various domains, such as education, health care, humanitarian response, farming, parenting, bodily modification, among many others.  Consider the current usage of the term ``hacking'' in the context of Internet companies in Silicon Valley, like Facebook and Google, who elaborate a ``hacker way'' and call what they do ``hacking'' instead of (or in addition to) coding, engineering, or entrepreneurship.  In public relation campaigns, hackers are described merely as ``doers'' since ``hacking just means building something quickly or testing the boundaries of what can be done''. %Citation?
According to this very generous definition, we can all be hackers, since we have been making artifacts for, at least, 2 millions years with the creation of our mode-1 stone tools.

--> add in here something about proliferation of hackfests, hackathons etc. 

These facts suggest that the figure of hacking and hackers has become somehow fundamental to a contemporary imaginary; but at the same time that it raises a question about what the anthropological study of hackers or hacking should look to in the future.  In this article we explore some of the challenges facing the anthropological study of hackers and hacking.  We suggest first of all that they are not necessarily the same thing:  the personhood and subjectivity involved in ``hacking'' is not continuous with the range of things that can be called ``hacks''---or put differenty not all hacks are perpetrated by hackers and not all hackers hack.   This strange configuration of personhood and practice is, we suggest, familiar to anthropology of technology generally, and also gives us a window onto contemporary mutations of power and knowledge today.

\section*{Stories of Hackers and Hacking}

From the early 1950's experimentation with communication and computing systems to the present-day hacker activist initiatives in the Global North and South, the discourse on hacking has been reconstituted by different genealogies, supporting different positionings with respect to whom and what counts as a legitimate expression of hacking.  Most of these influential narratives have been provided by journalists or hackers themselves, and not by scholars.  

The journalist Steven Levy is one of the most authoritive sources of this early history, published in his book \emph{Hackers} of 1984. Levy's work is examplary in offering early heroic stories of inventivity and exploration of computer systems based on life-histories, tracing the originns to the student group ``Tech Model Railroad Club'' (TMRC) of the Massachusetts Institute of Technology (MIT) of the 1950's. \emph{Hackers} has been translated in several languages and has been accepted among distinct hacker communities. Its narrative had the performative force of instituting a return to the figure of the ``virtuous hacker'' in the contemporary with its descriptions of the experience of early hackers around research centers at MIT and Stanford, as well as companies such as Apple and Sierra Games. Levy also popularized a positive definition of hacking as grounded in the ``hands-on imperative,'' the need for information freedom (as to facilitate technical exchange and promote further hacking), the rebelious attitude with respect to authority, centralization, and control of technical infrastructures, and the idea that technical work could be used to bring forth beauty and effect positive social change (Levy 1984).

A more recent account of the early origins of hacking was given by the technologist Phil Lapsley (2015) in his book ``Exploding the Telephone'' which reconstructed the early history of ``phone-phreaking'', that is, the precursor of computer hacking which targetted the exploration of phone systems. Lapsley's describe a genealogy which connects the early autnomist practice of Yippies of the 1960's counterculture movement with exploration of computing and electronics in the context of corporate control and centralization of computing infrastructures of the 1960's and 1970's.  Lapsley's account calls attention to a fundamental aspect of phone phreaking: a shared experience in which the telephone became the very embodiment of curiosity, and the phone network, a space for exploration, discovery, and socialization.  This is an important period of transition of the past 50 years in which we have witnessed the passage from large-scale, military-grade computer infrastructures to distributed personal computing power with massive network integration in the Euro-American world. The consequences of this transition are key for the study of digitization: the transformation of cultural practices through the usage of digital technologies, difference in infrastructure (of computing networks), materiality (of computing devices) and virtuality (of archival and social memories).

Hafner and Lyon, Where Wizards Stay up late. 

Raymond and auto-ethnography

A distinctive feature of hackers is their self-documentation---even in a few cases, their enthusiastic adoption of an ``anthropological'' mode to do so (sp. the work of Raymond).  Jargon File; (see also Michael Fischer, WOrlding Cyberspace).

Jason Scott's archival excess

Sterling, Hacker Crackdown; stories of Mitnck

Two important sociological and historiographic contributions in the literature depicting the rise and fall of the "hacker underground'' of the 1980's and 1990's: ``Hacker Culture'' by the communications scholar Thomas Douglas (2006) and ``Hackers'' by the sociologist Paul Taylor (1996). These two books are complentary in the sense that they describe the ``underground hacker scene'' of the United States and the United Kingdom in the period of popularization of hacker techniques and concomitant criminalization of its practice. Taylor's work is focused on the relationship between the nascent computer security industry of the 1990's with the hacker underground, which is a fruitful description of the duality which is characteristic of this lifeworld in which hackers are both the chaser and the chase, both on the side of law enforcement and on the side of the curious hacker collectives, navigating the thin line, most of the time indistinguishable, between the harmful and the inocuous, the virtuous and the vicious with respect to the impacts of the exploratory attitude of computer experts. Douglas' work is particularly useful in describing the discursive strategy in which the figure of the hacker as ``criminal'' is instituted -- having to create a mythology around the alleged superpowers of curious adolescents with access to a personal computer, a modem, and a phone line.


Canonical works that are generally about the ``culture'' of computing---and not always hacking specifically---include those of Sherry Turkle (Second Self, Life on the Screen), Diana Forsythe (Studying Those); the volume ``Cultures of Computing'' by Susan Leigh Star; the work of David Hakken; Lucy Suchman; Star and Ruhleder; Helmreich (Silicon Second Nature); Bryan Pfaffenberger.  Very little of this work is directly focused on hackers or hacking as such, but nonetheless constitutes some of the most significant work in English-language anthropology.

Benjamin Nugent on Nerds

Work in 2000s by Coleman, Golub, Kelty, Nardi, Boellstorff, Johns, Juris... 

Coleman and Golub, Genres, as an essential distinction. 

Coleman Ethnographies of the Digital

Debates about Gender;  James Leach and Dawn Nafus

Shifts towards questions of amateurism/expertise, participation and personhood. 

Yet another important piece of hacking history has been investigated by anthropologists
under the guise of the Free and Open Source technologies of the past thirty 
years. Two major contributions on this are ``Two Bits'' by Chris Kelty and ``Coding 
Freedom'' by Gabriella Coleman. The former provides the basis for the analysis of 
the historical formation of a new kind of publics which has a distinctive
charateristic of recursivity, that is, it is defined by a group of active
computer technologists whose means of organization relies on the need of openness and
possibility of redefining and reorganizing the very means of association. This
work has been important to highlight a new phase of hacking under the rubric of
Free Software as an emergent moral and technical order, as well as a new
mode of production of information technologies. Equally important for the
literature on hacking, Coleman's work emphasized the moral and legal
cultivation through which software developers of a particular Free Software 
project (Debian GNU/Linux distribution) are exposed in participating in
the collective effort of building an entire operating system by self-organizing
a large remote enterprise over the Internet. Both authors pioneered the 
investigation of the liberal underpinnings of Free Software as a technopolitical 
project.

Nicolas Auray

Sebastian Broca

The most recent publication on the topic of ``hacking'' has been published in France
by Michel Lallemand on the topic of ``hackerspaces'' and the rise of the
discourse and practice of ``making". Lallemant offered an ethnography of the
"maker movement'' in the San Francisco Bay Area with a focus on the question of
the transformations labor and its reorganization with the creation of independent
spaces for collaborative work with digital technologies. The book describes a 
community space for computing called ``Noisebridge'' which has been one of the most 
influential autonomous space for the ressurgence of the discourse on
hacking and self-organization for democratizing access to expert computing
knowledge around the globe. 


Albeit central to the study of alternative computer collectives outside the
United States and Western Europe, there are important gaps in this literature
to be addressed through an anthropology of computing which have to do with the 
study of the conditions for cultivation of expertise outside (and yet in 
connection with) the Euro-American context.

Yuri, Anita, Jenna Burrell, Silvia Lindtner, etc.?

\subsection{Hackers...}

Are hackers a kind of person?  The question only makes sense in an era when ``identity'' is an essential feature of personhood, and a collective of any size can be said to have a ``culture''--from neighborhood and office culture to national or religious culture.   There is a tendency to think of hackers first as a subculture of people with specific forms of affiliation, self-designation, and a sense of belonging or community.  But central to this identification of hackers as having an identity or community, is the existence of the ``hack.''  Hackers are said to have a culture of perpetrating a particular kind of act which can supposedly be distinguished from the actions of other kinds of people in other communities (especially, those in bureaucracies, hierarchies, corporations, or other hidebound organizations of the past).  The ``hack'' however, is mobile and re-usable, it often, but not always, takes the form of a tool or set of tools; it can be a one-off, but it is more often something that is re-usable.  Which is to say, the identity of ``hacker'' is troubled by the recognition that ``hacks'' are frequently borrowed, re-used, reconfigured and redeployed by people who only subsequently come to think of themselves as hackers---or maybe don’t do so at all. The diversity of kinds of people who get called hackers is a result of the fact that perpetrating a hack is the only thing necessary to make one a hacker: fundamental to the hacker identity (if such a thing exists) is the idea that your identity is irrelevant to your works.  If you pull off a great hack, other hackers will call you a hacker regardless of who you are. 

Such a complicated expression of personhood poses a dual problem for anthropological research.  On the one hand, it creates a challenge for making sense of personhood itself in an era where conventional idioms of identity are being contested in this way; on the other it deflects our attention to this act itself---``hacking''---and to the question of what it is and how it might be studied anthropologically.  Are the actions of  pirates, activists, criminals, cyber-warriors, and other such creatures called ``hacks''?  Why do we call mid-level engineers, librarians, scholars in universities or military professionals ``hackers''?   Do all of these people share a particular disposition of personhood, or do they share a milieu of technologies, material and institutional forms at their disposal?  How can we disentangle the sudden dispersal of hacking and hackers around the world?

Classically, there have been competing definitions of hackers.  Consider two distinctive definitions: one from the Internet engineering community and the other from the so-called ``hacker underground.''  For the former, a hacker is defined as a person ``who delights in having an intimate understanding of a technological system,'' whereas his or her opposite, a ``cracker,'' is an individual who attempts to access computer systems without authorization.  These hackers pride themselves on building complex systems out of available parts, or on getting around an engineering problem in a simple and elegant way.  As Coleman details, they are both highly individualistic, but oriented towards a community of other hackers (Coleman 2012).

By contrast, for underground hacker collectives, hacking can mean ``systems' penetration'' and exploitation as a  display of technical ability and mastery.  For them, hackers are people ``who gain unauthorized access to computer systems,'' a practice which is evaluated in terms of technical aptitude and virtue, measured up against the value of contributions in software code, information, and documentation.   ``Exploits'' become objects of value that circulate--at one time only for a kind of cultural capital amongst hackers, but increasingly today as part of a robust market for ``Zero-day'' exploits (Soghoian xxxx).   Increasingly, these forms of hacking are also well established in the military and defense world, both defensively and offensively.

Members of the early Internet engineering expert communities claiming the hacker identity would strongly disagree with the definition given by or of the hacker underground. Many of these communities were described in the origin stories of Free and Open Source development as a ``natural attitude'' of pioneer computer technologists.  They are strongly associated with research universities and with a certain mythical understanding of scientific method and practice; they point back to avatars such as the Digital Equipment Computer Users' Society (DECUS) in the sixties, or SHARE, the IBM mainframe computer users' group (Akera 20xx), the MIT community around the Tech Model Railroad Club, the Artificial Intelligence Laboratory operating system development staff, and the early community around Berkeley Unix System distribution, among many others.  The period which extends from the 1950's to the 1970's is identified in the literature with the work of early, pioneer hackers in pushing the boundaries of computing on many fronts: from hardware hacking to personal computers, from new operating systems to video-games and graphical user interfaces.

By contrast the ``hacker underground'' has its own mythical histories, associated with the history of phone phreaking, bulletin board services, magazines like 2600 and movies like \emph{War Games}.  These stories are more likely to reference the criminalization of hacking under the US Computer Fraud and Abuse Act, and the fabled exploits of people like Phiber Optik or the Cult of the Dead Cow.  While the Internet Engineers will point to Stephen Levy's \emph{Hackers} as canonical, the hacker underground will point to Bruce Sterling's \emph{Hacker Crackdown}.  Over the years, these two versions of hacking have competed and collaborated: they cross paths at conferences like DefCon, Chaos Computer Club, or HOPE, as well as at major open source and Free Software events.  The \emph{cause celebre} of intellectual property activism amongst hackers--Dmitri Sklyarov---was arrested at DefCon, but became a symbol for hacking as political form---''code is speech'' (Coleman 2009). 

\includepdf[pages={1,2},nup=1x2,scale=0.75]{images/Vitra_Hack}

\begin{figure}
  \centering
  \includegraphics[scale=0.6]{images/Workspirit13HACKStudioAKFB3}
  \includegraphics[scale=0.4]{images/HackDeskGrcic}
  \caption{The Hack Desk, designed by Konstantin Grcic, manufactured by Vitra.}
  \label{fig:hackdeskimage}
\end{figure}

Both of these communities reject with derision the widespread use of the term today in mainstream culture.  Facebook employees who ``hack'', ``brogrammers'' in Silicon Valley, and the generalized use of the term ``hack'' to mean ``do something'' are seen as corruptions of the tradition.   The Vitra ``Hack'' desk (See Fig), for instance, performs an attentuated and distorted version of hacking that emphasizes Silicon Valley neoliberal start-up culture, an ``individually adaptable private sphere,'' and a vision of flexibility in which the desks convert into sofas as part of a longed for collapse of work, leisure and political authenticity, primarily amongst white, upper middle class technology employees in Europe and America. 

But the mainstreaming of hackerdom is just as likely to expose the sexism and/or racism of past communities of hackers.  An ``elite hacker'' in an interview to an influential hacker zine declared the demise of the hacker underground with the xenophobic and mysoginistic observation that ``today it is claimed that the Chinese and even women are hackers.'' %citation
Symbolically violent in its own terms, this observation indexes not only the ethnocentric and gendered lifeworld of most hacker collectives, but the fact that hacking is no longer limited to the virtual play-fight among Anglo-American suburban adolescents, pointing also to questions of personhood with the evaluation of who gets to be considered a hacker.

What is important to emphasize in these examples is not the adoption of one definition or another, but the evidence of a series of disparate, conflicting, and generative differences for the contextualization the hacking. This is the evidence of an oscillation in respect to the value of ``hacking'' over time: fluctuating from positive to negative moral valences, that is, from elite, exclusive groups of technologists to online and offline communities marked by the rhetoric of opennness and transparency, or, from what the communications' scholar Douglas Thomas called the ``culture of secrecy'' of the hacker underground to a ``culture of collaboration, openness, and transparency'' as a shared utopian horizon (Kelty, Broca?).

\subsection{...or hacking?}

Is ``hacking'' a particular kind of action?  What differentiates it or gives it consistency today, and how might it be related to leaks, breaches, or other actions of information disclosure, circulation, both constructive and destructive?

Leaking, for instance, has become associated with hacking in the last decade. The furor around Wikileaks, for example, brought the fun-loving underground anti-collective Anonymous to world-wide attention via its coordinated attacks on corporations and governments.  Similarly, sharing or piracy: The May 15 movement in Spain was spearheaded alongside protests of La Ley Sinde---an anti-piracy law widely opposed by musicians and consumers alike, and ultimately merged with anti-austerity protests; key activists in the North African revolutions were also involved in the movements organized by Anonymous and were often connected to actors in both May15 and Occupy.  Occupy movements around the globe borrowed tactics, technologies, slogans and ideas that both explicitly and implicitly reference open source, free software, the pirate party and copyright reform movements—--all of this at the same time that US and Israeli spies were infecting Iranian nuclear power plants with the StuxNet virus (Kim xxx) , and the NSA and GHCQ were cataloguing all of this, and perpetrating some of the greatest ``hacks'' the internet has ever seen---and this we know only because of the even greater hack of Edward Snowden. 

Hacking, as a practice, is not confined to hackers, but increasingly a practice essential to the social fabric in surprising ways.  One might turn to ``practice theory'' in its various forms to explore this---are hackers a ``community of practice?''  Such a question would combine questions of personhood and practice in order to suggest that the identity of a hackers is based in the practice of hacking.  Indeed, such definitions are common (we ourselves have proposed them---in the case of a ``recursive public'' for instance Kelty 2008).  But this approach cannot accomodate the fact that hacking and hacks have become more obsteperous and unruly in their circulation: hackers fight hackers and trolls troll trolls and pirates steal from pirates today.  To subordinate personhood to shared practice would miss the mark. 

Instead, there are a diversity of ideologies supported by people who are called or call themsleves hackers. From the underground hacker collectives to ``black hat'' security researchers to spam-slinging criminal hackers, to the hard-core free speech and privacy cryptography defenders; from the dogmatic free software warriors to the pragmatic open source advocates, from the über-cool northern European art hackers to the goofy-but-terrifying Anonymous hackers, and so on.   As Coleman and Golub (2009) point out, there is no single liberal ideology that hackers adopt, but rather a range of ‘genres’ in which any given individual or group might operate, implying both the freedom and the constraints that word signals.

But what is the substance of these genres?  Are they just political interests, or are they something more?  We argue here that these ``genres'' are not about identity or collective belonging or community.  They are about practices to be sure, but they are bound up with new kinds of technological affordances that make practices less dependent on embodied skill than many other kinds of practice (e.g. glass blowing or cabinet making) and as a result more mobile and modifiable.  The expanded field of hacks, leaks, exploits, breaches, ops, campaigns and so on form the very substance of political life today.  

--> explain here that the anthropological reason to do this is to refine our ability to track and observe practices of power in ways that go beyond the simple Foucauldian model of sovereignth, biopower and exception.  To add to the repertoire of tools anthropologists themselves use in order to analyze power amongst different social actors.  

{  And especially: that it is about how techniques are recombined in a topology of power, not just the epochal shift of regimes of power.   It is this active ``recombination of elements'' that Foucault recognized as a way to stretch, transform or warp the patterns of correlation that make up a given topology of power. What better indication of this deformation of topology than the fact that today the pirates see themselves as having a coherent political platform while the social movement activists of Occupy claim not to.}


``Hacker practices'' don’t always originate with hackers, nor do they confine themselves to use by hackers. Tactics or practices are picked up by computer security scientists and researchers, security firms, police forces, political campaigns, anti-piracy outfits, analysts of all sorts, as well as spreading globally through networks of activists and legal firms, musicians and artists, or consumers and their pirates. Hackers go to work for Google or Facebook, anti-piracy companies appropriate the tools of hackers and pirates, and social movements look more and more like organizations than publics.  Power, in this sense, is about the recognition, appropriation, recomposition and redistribution of tactics and practices: not ideologies or genres—but recomposable instances of power.  The technique of the DDOS attack, for instance, or the use of a cease and desist letter, the creation of commons-producing copyleft licenses, or the use of the bit-torrent protocol are all practices of hacking, but they are not all the same kind of thing.  Building on the work of Coleman and Golub (2009) we offer a brief proposal for a set of distinctions that might make sense of these differences, and generalize the generic categories they offer in order to expand the analytic possibilities.  The first three practices listed here (invention, inversion and figuration) are quasi-direct forms of action; they are accessible to anyone, even if the barriers are significant, and do not require significant investments of money or organization to achieve.  The last two (regulatory action and force), however, are often more second-order or ``representative'' in that they often require physical, financial or organizational resources and a certain scale and depth of involvement to perform. 

\subsubsection{Invention}

Invention includes building and making things, and not only material things, but so-called ``immaterial'' products such as software or organizations.  Invention is intended to cover those practices that arise out of a lack or out a sense of need of some sort; where there is no clear need, there is nonetheless a recognition of possibility based in the existence of multiple existing tools and components.  So creating software, hacking together hardware components or starting up an organization are all practices of invention in so far as they are carried out to solve a problem or respond to a lack that makes certain possibilities clear.  

Invention as we use it here implies a material culture of ready-made, accessible, often lego-like parts---and not the fabled \emph{de novo} creation of a technology or idea.  It implies easy access to tools (recall the ethic of the Whole Earth Catalog: ``Access to Tools'' described by Turner 2006), mutual aid and instruction and an appreciation for both the DIY possibilities of our world, and sometimes a respect for (and desire to contribute to) the coordinated engineering necessary to bring them into being.

Furthermore, the practice of invention implies an affinity based in shared understandings of how things work.  Whether that is a classic engineering culture (based in University and/or corporate practice), a DIY geek culture, a UNIX culture, a Windows culture, a glass-blowing culture (O’Connor) or whatever, invention demands not only skill but the capacity to recognized others with more or less the same skills, habits of practice and commitments to certain kinds of technological or material choices (Sennet on the Craftsman).

\subsubsection{Inversion}

Inversion is a term meant to signal a practice that---at least under the label of ``hacking''---is often insufficiently distinguished from invention.  Inversion is the kind of hacking that involves finding and exploiting vulnerabilities, using systems against themselves, inverting the intended purpose of a system, or remixing one for purposes of critique, parody, ridicule or something more practical (Galloway and Thacker).  In terms of hardware it includes the practices of modding and customization; in terms of software in includes the practice of finding and exploiting weaknesses in software (for good or evil intent) or recombining software elements in a surprising and clever way in order to achieve a new goal.  In legal terms the most famous ``inversion'' is the GPL free software license, which uses existing statutory copyright law to accomplish something it was not intended to do.  Wikileaks is in some ways a practice of inversion, insofar as it makes public information not intended to be. 

Inversion implies institutional structures or infrastructures with a certain transparency (one must be able to see how something works in order to exploit it).  Or where that transparency is absent, at least some route towards reintroducing it (leaking documents, reverse engineering, etc). The debates about secrecy and transparency are very often deeply engaged in a particular form of inversion---but as we will see, also engage in the tactic of figuration.  Inversion also implies faith in engineering and in the rule of law: for something to be inverted is not the same as for it to be destroyed.  GPL licenses work because copyright law is legitimate and enforced.  Remix or recombination of software is done because the challenge is to make something work better or meet higher standards of practice.  By extension, the faith is that things can always be and become better—only imperfect things can be hacked, inverted, in this sense.

Similarly inversion implies patterns of regularity that can be exploited---the very thing upon which the practice of invention often depends as well.  Inversion works upon certain technological or organizational preferences shared amongst a large group of people—for instance the use of SCADA control systems amongst process engineers who design large industrial plants allows for hackers to imagine exploits with powerful effects on the material world.   Similarly, entrenched social behaviors are often exploited in ``social engineering'' hacks that rely on the regularities of organizational design and human behavior.

\subsubsection{Figuration}

Figuration is a term meant to capture the more traditional and recognizable forms of political action: rhetorical persuasion, ideological argument, political advocacy, etc.  Classic descriptions of the functions of the public sphere tend to characterize it as an issue of sovereignty---the ability of ``the people'' to speak to and force changes amongst established domains of power like the state, the church, the military etc. (B. Anderson).  All of the tactics involved here are about visibility and the legitimacy of a political process.  In eras or locations where there is nothing like a legitimate democratic government, tactics of figuration are often ineffective or irrelevant. 

Materially speaking, figuration also depends on shared communication structures---and this is where the overlap with the tactics of invention or inversion are most intriguing.  When communication practices are invented or inverted in response to weak practices of figuration.  Longstanding institutional structures (such as public hearings or requests for comments) both facilitate and provide friction on the practice of figuration.  Figuration also implies a stable set of narratives---what Charles Taylor dubbed social imaginaries---against which it is possible to make arguments, whether rational or emotional, which have political efficacy. 

-->> include here EFF, Creative Commons, SOPA/PIPA, --- maybe Tor and Internet defense league as example of invention for the defence of figuration.  

Traditional institutional structures of government, such as new political parties, as well as emergent ones are focused on figuration. Non-governmental organizations engaged in policy analysis or advocacy, social movements engaged in protest, or scholars engaged in criticism and research all work at the level of figuration, in part because it has traditionally worked at the largest scale—that of the public sphere or civil society.

Substantively (at least in the domains of intellectual property and information technology) practices of figuration have been heavily focused on issues such as privacy, transparency, free speech, freedom to operate (or innovate) and network neutrality.   These issues subtend a long and rich discourse that includes both scholarly debates and conventional understandings of these concepts and their value to our lives. 

\subsubsection{Regulatory action}

Regulatory action includes any form of policy change intended to introduce, maintain or extend control.   State-based forms of regulatory action are the most obvious and familiar but large corporations also engage in regulatory action of particular kinds routinely---especially those industries who control networks or technologies in widespread use. 

Regulatory action implies large bureaucratic organizations of a classic Weberian type---rule based, hierarchically ordered, and subject to regimes of oversight and transparency.  As a result, regulatory action is relatively rare, and comparatively complicated to carry through.  The tactics of invention, inversion and figuration are often oriented towards influencing, responding to or disrupting regulatory action of various kinds---the 2012 case of protests against SOPA/PIPA being a clear case; another case would be Operation Payback conducted by members of Anonymous against the global credit card companies Mastercard and Visa, who had engaged in the regulatory action of systematically blocking PayPal donations to the Wikileaks organization. 

Regulatory action also implies a professional cadre acquainted with the rules and procedures of these organizations, oriented towards professional norms of democratic governmental practice, even if in practice such norms are honored in the breach or routinely subjected to the power of external interests.  Nonetheless, regulatory action is something that is routinely pursued strategically by different groups---a corporation lobbies congress, one groups sues another or asks a court for an injunction; lawyers send letters; states create new roles and appoint people to them (e.g. IP Czar); courts mete out punishment etc.

Inversion often borders on regulatory action when it is used strategically to achieve something in the interests of a particular entity, but so too does figuration, which is often the tactic most often employed to support or protest a proposed legal change (both were used in the case of SOPA/PIPA).  Cases of significant interest include those where regulatory action look more like cases of inversion---i.e. where a legal action, for instance, is used to threaten, intimidate or censor a particular group. The injunctions filed against MegaUpload and Kim Dotcom for instance, are represented as mere enforcement of the law, but in reality represent the effective organization of state power by industry organizations like the MPAA and the RIAA. 

\subsubsection{Enforcement(s)}

Finally, there are the practices of enforcement. Often only implicitly or metaphorically included in a Foucauldian analysis (discipline is held to be more insidious and more profound kind of force, a ‘government of the soul’ for instance).  While classic displays of sovereign power are relatively rare (helicopter raids by anti-terrorist forces on New Zealand mansions of flamboyant wannabe hackers notwithstanding), they remain a central tactic in the repertoire of power, and they are by no means 
restricted to a state and its soldiers and police forces. 

Many forms of enforcement are widely available today.  A range of tactics and practices of which the DDOS is only the most common are not confined to any particular segment of society, but available to anyone who might, for instance, download and install a copy of the Low Orbit Ion Cannon.   Legal tools like cease and desist letters are also routinely used as a tactic of force; even patent litigation can be figured this way when conducted by so-called patent trolls.  Networked-based forms of disruption are very often simultaneously tactics of invention or inversion—coming into existence in response to the actions of states or corporations.  Piracy and cracking generally might be said to move from being a tactic of inversion when a vulnerability in a copy-protection scheme is merely demonstrated (e.g. Sklyrov demonstrating holes in the eBook Reader) to a tactic of force when that vulnerability is routinely exploited.




\begin{figure}
  \centering
  \includegraphics{images/protocolstackV2}
  \caption{Protocol Stack of Power: an analytic decomposition of hacking practices and how they might relate}
  \label{fig:ProtocolStack}
\end{figure}


How do all these practices fit together?  As a ``topology'' perhaps, in which different techniques emerge in response to others, and to events and thresholds, both political and technological.  Inventions and inversions often come in the face of regulatory actions and their enforcement -- the case of Napster and Bittorrent in response to the DMCA for instance; or the figuration of hacking in response to the CFAA; or the inversion of the GPL in response to the enformcement of conventional IP law. 

Figure x suggests that hacking today takes the form of stack of pratices available to different actors at different times, but responding to specifical technological and political affordances. 

Question: how does the personhood of hackers (of different sorts) connect to this stack of practices?  In what ways can we understand the ethical cultivation of particular kinds of skills (skills of invention, inversion etc. vs. those of figuration or regulation) as leading to different forms of personhood?  Does this topology remain stable?    The question we end with concerns the dynamism of this topological stretching, and the production of stability.  Just because the contemporary moment is one of exacerbation and failure does not imply that we have entered the epoch of uncertainty—rather the question is: constant oscillation or stabilizing configuration.  History favors the stabilization of forms of power, but the manifest speed and ease with which the contemporary topology can be stretched and deformed suggests a perpetual oscillation or turbulence, limited only by the energy and enthusiasm that can be committed to the various practices of invention, inversion, enforcement or regulation.  


\section{Hacking and Digitization: Open Questions}

{\bf Imperial imperative of digital technology} to transform sociotechnical 
difference through contact into its ressemblance (or poorly made copy). Anthropological
studies of the global circulation of digital technologies and expert technologies
have demonstrated the naturalization of Euro-American underlying assumptions in
digital design: from user interfaces to data models (Downey 199x); assumptions
of usefulness of particular technologies based on their place of origin (Yuri, 20xx); 
and the imposition of particular projects from centers of production to disconnected
peripheries of the global South (Chan 200x).

% --> I would add here: how is the global difference in personhood of hackers related to the global accessibility of hacks?  What are examples of the reverse process, where the actions of global hackers (say, book pirates in russia or activists in Tunisia, or Anonymous' attacks on ISIS) have an impact on mainstream practices?

{\bf Myth of Autonomous Technology} which pressuposes its separation from human societies.
Pfaffenberger has pioneered the anthropological study of sociotechnical systems with a
refusal to assume the separation between technology and society. In his studies of digital
technology (the Usenet and the Personal Computer, for instance), he has argued for the 
processual nature of technological design as invariably embedded in cultural systems. 
This approach is particularly useful for the study of digitization for providing a 
general guideline in studying how different collectives arrange the relationship between 
technological artifacts and infrastructures vis-a-vis cultural institutions. In respect 
to hacking in particular, Pfaffenberger's anthropology of technology serves as a critique
of the big divide between technology and society by addressing the ways in which 
digital technologies are politicized and collectives are computerized and cultural practices
are digitized at once. That is, how artificial languages and technical abilities become key 
competences of a hacker embodiment, therefore hybridizing the technical and the cultural, 
digital and the moral orders.

%-->  I would move this to the lit review section... because I think it's actually background for our attempt here to identify distinctions amongst the practices/genres of hacking. 
 
{\bf Power imbalances, gender and ethnic troubles.} One of the key markers of difference 
among computer experts concerns gender and the exercise of particular forms of 
masculinity in heated arguments around technical issues. It is key for displaying 
structured relations of power and practices of value and worth assessment within 
hacker communities.  Questions of power imbalance and difference refer to the 
nature of sociotechnical ties within expert communities.

Hacker collectives imagine themselves as meritocratic orders for having instances 
in which merit is recognized and rewarded through vetted collective mechanisms, 
but it is also oblivious to unequal starting points for the development of tech 
skills for participation in hacker collectives, such as Free and Open source 
projects. The gender question in digital communities is open for empirical study since
most of the literature on hacking has dedicated, at most, footnotes to it. The question 
of power imbalance (including socioeconomic and ethnic differences) is a broader issue 
concerning the place and role of hackers in various domains of computing: hobbyist, 
academic, corporate.

%--> so is this about a "power imbalance" really?  To me it seems to be more about inequality and injustice than about power.  It's just as likely that gender inequity exists amongst powerful people (Silicon Valley venture capitalists) as amongst the weakest (local bike cooperative in my neighborhood).  So I don't see this as an "open question"--- or at least, it's open only in the sense that we don't address it in any novel way here...


{\bf Informational and intellectual enclosures.} Another important aspect of hacker politics
has to do with the struggle against informational enclosures. The rebelious attitude of most
hacker collectives with respect to hoarding of information and computational resources stems
from pioneer hacker collectives, for whom the scarcity led them to exploit available systems
and force their access onto computer laboratories (that were heavily guarded, given their
absurd cost of ownership at the time). The struggle against the monopolization and control of 
computing has advanced in many areas -- such as game piracy, information security, and 
Free and Open Source software development -- creating communities whose members are 
identified as hackers but whose practices differ in respect to the moral and technical 
orders they inhabit. One of the most vocal hacker community in the dispute against the 
advancement of the Intellectual Property regime (IPR) is the Free Software community, which 
has consistently for the past 30 years created mechanisms, both legal and technical, to subvert 
the logic of copyright and create alternatives to the patent systems in many legislations
across the world.

% -->  I feel like this one is in line with the five practices above, and we could describe it in those terms:  IP Law and Invention are counteracted by the "inversion" of the GPL and the "figuration" of an informational commons... DMCA is a "regulatory response"-- jailing Sklyarov is one kind of enforcement, creating BitTorrent is an invention in response to the crackdown... etc.  

{\bf Algorithmic Agencies.} For anthropological studies of digital platforms and cultures,
the ethnographer must attend to the duality with respect to the submission of users to
algorithmic agency and the fact that certain human actors are active agents in what has
been called the ``web of computing'' (Scacci and Kling) which includes many layers of 
subsystems that compose the domain of computing as a whole, not as a discrete-entity type). 
Anthropology is well suited to study algorithmic agencies with ethnographic approaches to 
the lifeworld of research and development laboratories where computer systems are 
designed, discussed, re-designed, and implemented. Important contributions
in this area have given by the pioneer ethnographer Diana Forsythe in her study with AI 
researchers building ``expert systems'' for the medical field, as well as Greg Downey in his
work with engineers working with Computer-Aided Design technologies, Harry Collins in his
sociology of technical expertise with the study of AI-based expert systems, and Lucy 
Suchman with interactional domain of experience involving engineers and machines at Xerox. 

%-->  too far afield I think.  There is room to ask, however, "how is what we are talking about here---hacking as personhood and practice---related to interests in big data and algorithms (e.g. Kate Crawford, Tarleton Gillespie, Natasha Schull, Frank Pasquale etc).

{\bf Collaborative Horizons.} The digital is not a panacea for the problems of collaborative 
research in anthropology and the human sciences at large, but it can certainly help 
if digital technologies are re-designed to futher promote collaborative engagements 
between ourselves and ourselves with the communities we study with. Digitization of 
field research carries potential benefits with respect to the ease of archival and 
data sharing, allowing for longitudinal and extensive compartive studies. But is also 
carries serious issues of data privacy and anonymity as most researchers (in both the 
sciences and humanities) are not well equipped and informed to handle problems of 
information security. Hacking, in this regard, is an important source of ``best practices'' 
to deal with questions of information security, but also sharing and remote collaboration across transnational lines of exchange.

%-->  This is a good thing to conclude with--- essentially, "are anthropologists hackers, should we be?"



\section*{Conclusion}


%\bibliography{hack_entry} 
%\bibliographystyle{agsm}

\printbibliography 
\end{document}
